%\begin{center}
%\large \bf \runtitulo
%\end{center}
%\vspace{1cm}
\chapter*{\tituloTesis}

En este trabajo se estudia la posibilidad de demostrar la ejecución de programas de alto nivel utilizando Plonky2, un sistema de pruebas criptográficas de tipo ZK-SNARK que no requiere de un \textit{trusted setup}. El trabajo se enmarca en el área de las pruebas de conocimiento cero (Zero Knowledge Proofs), presentando una introducción a las primitivas matemáticas y criptográficas necesarias para comprender estos protocolos, así como un análisis superficial de los sistemas relevantes como PLONK.

Se propone y desarrolla una herramienta que traduce programas de ACIR (la representación intermedia de Noir) a Plonky2. De esta manera, se logra la construcción de pruebas verificables de ejecución en entornos con recursos limitados de almacenamiento, superando una de las principales restricciones de Barretenberg.

El documento incluye la descripción formal de la traducción de primitivas y operaciones, una evaluación experimental del desempeño obtenido y un análisis comparativo en términos de tiempos de ejecución y tamaños de artefactos generados. Finalmente, se discuten los resultados alcanzados y se plantean líneas de trabajo futuro. 

\bigskip

\noindent\textbf{Palabras clave:} Pruebas de conocimiento zero, zero knowledge proofs, ZK, SNARK, PLONK, Plonky2, Noir, ACIR, Barretenberg, Criptografía.