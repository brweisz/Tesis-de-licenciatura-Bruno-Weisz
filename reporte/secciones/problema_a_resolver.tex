\chapter{Problema a resolver}\label{sec:problema}
Noir es un lenguaje de programación de alto nivel que compila a una representación intermedia llamada ACIR y permite generar pruebas de ejecución usando un prover llamado Barretenberg. Barretenberg usa KZG como Polynomial Commitment Scheme, por lo que requiere un CRS que puede crecer arbitrariamente en tamaño; por ende, necesita una ceremonia con varios agentes involucrados y espacio de almacenamiento suficiente para generar las pruebas. La propuesta de la tesis es adaptar Noir para que pueda ser usado con un prover llamado Plonky2, el cual utiliza FRI en lugar de KZG y por ende no tiene los problemas anteriores. 

Parte del trabajo consiste en investigar ambas herramientas con la profundidad suficiente para poder implementar y explicar la herramienta generada, por lo tanto el desarrollo contará con algunas secciones dedicadas a indagar en Plonky2 y Noir. El grueso del trabajo consiste en el desarrollo de la herramienta, justificando las decisiones de diseño tomadas y brindando desarrollos matemáticos suficientes para demostrar la correctitud de las soluciones. Además, para verificar que la herramienta cumple con su propósito, el código será acompañado de un conjunto exhaustivo de tests. 

Finalmente, haremos un reporte de la performance de la herramienta, midiendo magnitudes como tiempos de ejecución o tamaños de artefactos generados que se detallarán en la sección \ref{sec:materiales_y_metodos}. Por último, se realizará una comparación entre los tamaños de las pruebas de Plonky2 con el CRS mínimo que Barretenberg necesita para algunos casos, para evaluar si esta alternativa presenta alguna ventaja espacial o no. 